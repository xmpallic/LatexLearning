\documentclass[UTF8]{beamer}

\usepackage{ctex}

\begin{document}
	
\begin{frame}
	\title{Beamer使用}
	\author{Xmp}
	\date{\today}
	\maketitle
\end{frame}
	
\begin{frame}{这里是标题。}
	这里是正文。
\end{frame}

%\frame{\tableofcontents}%列目录之前需要设置节和小节。
%\frame{\tableofcontents[currentsection]}%显示当前的进程。

\begin{frame}{部分说明}
	普通文档中的很多内容也都可以用于幻灯页,比如第二章里的列表和盒子,第五至八章里的各种插图,第九章里的表格。有时为了平衡页内空白,也可以使用第十章里的双栏。这些内容在幻灯和普通文档中的使用方法相似,此处不赘述。\\
	
	Beamer自己也提供了一些比较适用于幻灯的环境,比如下面示范的几个矩形块环境。它们都是在一个矩形块内输出一段文字,那个参数就是一个标题,它们的差别只是标题的颜色。
\end{frame}

\begin{frame}{测试2}
	\begin{block}{第一个盒子}
		第一个盒子中的内容。
	\end{block}

	\begin{alertblock}{不同的盒子}
		不同盒子的内容。
	\end{alertblock}

	\begin{exampleblock}{第二个不同的盒子}
		第二个不同盒子的内容。
	\end{exampleblock}
\end{frame}

%分步展现,使用\pause命令,下面的代码实际上生成了四页PDF,依次重叠。
\begin{frame}{使用特效,分步展现}
	\begin{itemize}
		\item 第一页
		\pause
		\item 第二页
		\pause
		\item 第三页
		\pause
		\item 第四页
	\end{itemize}
\end{frame}
\end{document}