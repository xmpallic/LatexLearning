\documentclass[UTF8]{article}

\usepackage{ctex}
\usepackage{booktabs}
\usepackage{listings}
\usepackage[table]{xcolor}
\usepackage{xcolor}
\usepackage{tabularx}
\usepackage{warpcol}
\usepackage{multirow}
\usepackage{longtable}
\usepackage{rotating}
\usepackage{colortbl}

\title{表格}
\author{Xmp}
\date{\today}

\begin{document}
\maketitle
\setcounter{tocdepth}{7}
\tableofcontents

\begin{abstract}
	制作表格的简单步骤。
\end{abstract}

\section{简单表格}
tabular环境提供了最简单的表格功能。它用 \textbackslash hline 命令表示横线,|表示竖线;用 \& 来分列,用 \textbackslash \textbackslash 来换行;每列可以采用居中、居左、居右等横向对齐方式,分别用 l、c、r 来表示。

\begin{tabular}{|l |c|r|}
	\hline
	操作系统 & 发行版 & 编辑器 \\
	\hline
	Windows & MikTex & TexMakerX \\
	\hline
	Unix/Linux & teTex & Kile \\
	\hline
	Mac OS & MacTex & TexShop \\
	\hline
\end{tabular}

表格简化为科技文献中常用的三线表。下面例子中的三条横线一样粗细,如果嫌它不够美观,可以使用booktabs宏包。三条横线就分别用\textbackslash toprule、\textbackslash midrule、\textbackslash bottomrule等命令表示。改进后的表格见下面。
\begin{table}[htbp]
\centering
\begin{tabular}{lll}
	\toprule
	操作系统 & 发行版 & 编辑器 \\	
	\midrule
	Windows & MikTex & TexMakerX \\
	Unix/Linux & teTex & Kile \\
	Mac OS & MacTex & TexShop \\
	\bottomrule
\end{tabular}
\end{table}

tabular 环境中的行可以采用居顶、居中、居底等纵向对齐方式,分别用 t、c、b 来表示,缺省的是居中对齐。列之间的分隔符也可以改用其他符号,比如用 || 来画双竖线。

\lstset
{
	numbers=left, 
	numberstyle= \tiny, 
	keywordstyle= \color{ blue!70},
	commentstyle= \color{red!50!green!50!blue!50}, 
	frame=shadowbox, % 阴影效果
	rulesepcolor= \color{ red!20!green!20!blue!20} ,
	escapeinside=``, % 英文分号中可写入中文
	xleftmargin=2em,xrightmargin=2em, aboveskip=1em,
	framexleftmargin=2em
}
\begin{lstlisting}
语法:[纵向对齐]{横向对齐和分隔符}
\end{lstlisting}


\section{宽度控制}
有时我们需要控制某列的宽度,可以将其对齐方式参数从 l、c、r 改为p{宽度}。这时纵向对齐方式是居顶,t、c、b 等参数失效。
\begin{table}[htbp]
\centering
\begin{tabular}{p{80pt}p{80pt}p{80pt}}
	\toprule
	操作系统 & 发行版 & 编辑器 \\	
	\midrule
	Windows & MikTex & TexMakerX \\
	Unix/Linux & teTex & Kile \\
	Mac OS & MacTex & TexShop \\
	\bottomrule
\end{tabular}
\end{table}

使用宽度控制参数之后,表格内容缺省居左对齐。我们可以用列前置命令 >\{\} 配合 \textbackslash centering、\textbackslash raggedleft命令来把横向对齐方式改成居中或居右。列前置命令仅对紧邻其后的一列有效,其语法如下:
\begin{lstlisting}
	语法:>{命令}列参数
\end{lstlisting}
%\begin{table}[htbp]
%	\centering
%	\begin{tabular}{p{80pt}>{\centering}p{80pt}p{80pt}}
%		\toprule
%		操作系统 & 发行版 & 编辑器 \\	
%		\midrule
%		Windows & MikTex & TexMakerX \\
%		Unix/Linux & teTex & Kile \\
%		Mac OS & MacTex & TexShop \\
%		\bottomrule
%	\end{tabular}
%\end{table}

若要控制整个表格的宽度,可以使用 tabularx  的同名环境,其语法如下,其中X参数表示某列可以折行。
\begin{lstlisting}
语法:{表格宽度}{横向对齐、分隔符、折行}
\end{lstlisting}
\begin{table}[htbp]
	\centering
	\begin{tabularx}{350pt}{lXlX}
	\toprule
	李白 & 平林漠漠烟如织,寒山一带伤心碧。暝色入高楼,有人楼上愁。
	玉阶空伫立,宿鸟归飞急。何处是归程,长亭更短亭。&
	泰戈尔 & 夏天的飞鸟,飞到我的窗前唱歌,又飞去了。
	秋天的黄叶,它们没有什么可唱,只叹息一声,飞落在那里。\\
	\bottomrule
	\end{tabularx}
\end{table}

如果想把纵向对齐方式改为居中和居底,可以使用array宏包,它提供了另两个对齐方式参数:m\{宽度\}、b\{宽度\}。

\section{跨行跨列}
有时表格某单元格需要横跨几列,我们可以使用\textbackslash multicolumn命令,同时使用booktabs宏包的\textbackslash cmidrule 命令来画横跨几列的横线。它们的语法如下:
\begin{lstlisting}
语法:\multicolumn{横跨列数}{对齐方式}{内容}
语法:\cmidrule{起始列-结束列}
\end{lstlisting}
\begin{table}[htbp]
\centering
\begin{tabular}{lll}
	\toprule
	& \multicolumn{2}{c}{常用工具} \\
	\cmidrule{2-3}
	操作系统 & 发行版 & 编辑器 \\	
	\midrule
	Windows & MikTex & TexMakerX \\
	Unix/Linux & teTex & Kile \\
	Mac OS & MacTex & TexShop \\
	\bottomrule
\end{tabular}
\end{table}

跨行表格可以使用 multirow 宏包的\textbackslash multirow 命令,其语法如下,
\begin{lstlisting}
语法:\multirow{竖跨行数}{宽度}{内容}
\end{lstlisting}
\begin{table}[htbp]
\centering
\begin{tabular}{lllc}
	\toprule
	操作系统 & 发行版 & 编辑器 &用户体验\\	
	\midrule
	Windows & MikTex & TexMakerX &
	\multirow{3}{*}{\centering text}\\
	Unix/Linux & teTex & Kile \\
	Mac OS & MacTex & TexShop \\
	\bottomrule
\end{tabular}
\end{table}


\section{数字表格}
当表格中包含大量数字时,手工调整小数点和数位的对齐很麻烦,这时可以使用warpcol宏包。它为 tabular 环境提供了一个列对齐参数 P,其语法如下,其 m和 n 分别是小数点前后的位数,数字前的负号可选。
\begin{lstlisting}
语法:P{-m.n}
\end{lstlisting}

\begin{table}[htbp]
\centering
\begin{tabular}{P{2.5}P{-2.5}}
	\toprule
	\multicolumn{1}{c}{数学常数} &
	multicolumn{1}{c}{物理常数} \\
	\midrule
	3.14159 & 2.99792 \\
	27.18281 & -17.58819 \\
	\bottomrule
\end{tabular}
\end{table}
使用 multicolumn 命令是为了保护表头,防止它们被 P 参数误伤。把跨列命令的列数设为1是设置单元格格式的一种常用方法。


\section{长表格}
使用longtable宏包。步骤如下:\\
1. 首先用 longtable 环境取代 tabular 环境;\\
2. 然后在表格开始部分定义每页页首出现的通用表头,表头最后一行末
尾不用 \textbackslash \textbackslash 换行,而是加一个 \textbackslash endhead 命令;\\
3. 接着定义首页表头 (如果它和通用表头不同的话) ,同样地最后一行用\textbackslash endfirsthead 命令结尾;\\
4. 然后是以 \textbackslash endfoot 命令结尾的通用表尾;\\
5. 然后是以 \textbackslash endlastfoot 命令结尾的末页表尾 (如果它和通用表尾不同的话) ;\\
6. 最后是表格的具体内容。\\
这\\里\\加\\些\\东西,以便跨页。

\begin{longtable}{ll}
	\multicolumn{2}{r}{接上页} \\
	\toprule
	作者 & 作品 \\
	\midrule
	\endhead
\caption{长表格}\\
	\toprule
	作者 & 作品 \\
	\midrule
	\endfirsthead
	\bottomrule
	\multicolumn{2}{r}{接下页\dots}\\
	\endfoot
	\bottomrule
	\endlastfoot
	白居易 & 汉皇重色思倾国,御宇多年求不得。\\
	 & 杨家有女初长成,养在深闺人未识。\\
	& 天生丽质难自弃,一朝选在君王侧。\\
	 & 回眸一笑百媚生,六宫粉黛无颜色。\\
	& 春寒赐浴华清池,温泉水滑洗凝脂。\\
	 & 侍儿扶起娇无力,始是新承恩泽时。\\
	& 云鬓花颜金步摇,芙蓉帐暖度春宵。\\
	 & 春宵苦短日高起,从此君王不早朝。\\
	& 承欢侍宴无闲暇,春从春游夜专夜。\\
	 & 后宫佳丽三千人,三千宠爱在一身。\\
	& 金屋妆成娇侍夜,玉楼宴罢醉和春。\\
	 & 姊妹弟兄皆列土,可怜光彩生门户。\\
	& 遂令天下父母心,不重生男重生女。\\
	 & 骊宫高处入青云,仙乐风飘处处闻。\\
	& 缓歌慢舞凝丝竹,尽日君王看不足。\\
	 & 渔阳鼙鼓动地来,惊破霓裳羽衣曲。\\
	& 九重城阙烟尘生,千乘万骑西南行。\\
	 & 翠华摇摇行复止,西出都门百余里。\\
	& 六军不发无奈何,宛转蛾眉马前死。\\
	 & 花钿委地无人收,翠翅金雀玉搔头。\\
	& 君王掩面救不得,回看血泪相和流。\\
	 & 黄埃散漫风萧索,云栈萦纡登剑阁。\\
	& 峨嵋山下少人行,旌旗无光日色薄。\\
	 & 蜀江水碧蜀山青,圣主朝朝暮暮情。\\
	& 行宫见月伤心色,夜雨闻铃断肠声。\\
\end{longtable}


\section{宽表格}
使用rotating宏包,用sidewaystable环境代替table环境即可。
\begin{sidewaystable}
	\caption{主流英文词典}
	\label{tab:dict}
	\centering
	\begin{tabularx}{550pt}{Xllcrrr}
		 \toprule
		Title & Abbr & Publisher & Year & Pages & Entries &
		Price \\
		 \midrule
		Oxford English Dict, 2nd Ed & OED & Oxford Univ
	& 1989 & 21,728 & 616,500 & 995 \\
		\midrule
		 Shorter Oxford English Dict, 7th Ed & SOED & Oxford
		Univ
		& 2007 & 3,888 & 600,000 & 175 \\
		 New Oxford Dict of English, 2nd & NODE & Oxford Univ
		& 2005 & 2,112 & 355,000 & 68 \\
		 Webster's Third New International Dict & W3 & Merriam-
		Webster
		& 1961 & 2,816 & 476,000 & 129 \\
		 American Heritage Dict, 4th Ed & AHD & Houghton
		Mifflin
		& 2000 & 2,112 & 90,000 & 60 \\
		 Random House Webster's Unabridged Dict, 2nd Ed &
		Random & Random House
		& 2005 & 2,256 & 315,000 & 69 \\
		 \midrule
		Concise Oxford Dict, 11th Ed & COD & Oxford Univ
		 & 2006 & 1,728 & 240,000 & \\
		Chambers Dict, 10th Ed & Chambers & Chambers Harrap
		 & 2006 & 1,872 & & 50 \\
		Collins English Dict, 9th Ed & Collins & HarperCollins
		 & 2007 & 1,888 & & 67 \\
		Longman Dict of Contemporary English, 4th Ed & Longman
		& Longman
		 & 2005 & & 207,000 & 71 \\
		Merriam-Webster's Collegiate Dict, 11th Ed & & Merriam
		-Webster
		 & 2003 & 1,664 & 225,000 & 26 \\
		American Heritage College Dict, 4th Ed & & Houghton
		Mifflin
		 & 2007 & 1,664 & & 26 \\
		Random House Webster's College Dict & & Random House
		 & 2005 & 1,632 & & 26 \\
		Webster's New World College Dict, 4th Ed & & John
		Wiley \& Sons
		& 2004 & 1,744 & 160,000 & 26 \\
		\bottomrule
		\end{tabularx}
\end{sidewaystable}

\section{彩色表格}
使用colortbl宏包,它提供
的 \textbackslash columncolor、\textbackslash rowcolor、\textbackslash cellcolor 命令可以分别设置列、行、单元格的颜色。这三个命令的基本语法相似:
\begin{lstlisting}
语法:{颜色}
\end{lstlisting}
\textbackslash columncolor需要放到列前置命令里,rowcolor、\textbackslash cellcolor 分别放到行、单元格之前。colortbl 宏包可以使用 xcolor 宏包的色彩模型;两者同时,前者不能直接加载,需要通过后者的选项 table 来加载。三个命令同时使用时,它们的优先顺序为:单元格、行、列。	
\begin{table}[htbp]
	\centering
	\caption{colortable}
	\begin{tabular}{l>{\columncolor{yellow}}ll}
		\rowcolor{red}操作系统 & 发行版 & 编辑器 \\
		Windows & MikTeX & TexMakerX \\
		\rowcolor{green}Unix/Linux & \cellcolor{red}teTeX
			& Kile \\
		Mac OS & MacTeX & TeXShop \\
		\rowcolor{blue}通用 & TeX Live & TeXworks \\
	\end{tabular}
\end{table}

xcolor宏包的rowcolors 命令 (需要 colortbl 宏包的支持) 可以分别设置奇偶行的颜色。该命令语法如下:
\begin{lstlisting}
	语法:{起始行}{奇数行颜色}{偶数行颜色}
\end{lstlisting}
下面代码第 14 行的 \textbackslash hiderowcolors 命令是用来暂停显示前面设置的奇偶行颜色,否则后面的其他表格会继续显示颜色。另一个命令\textbackslash showrowcolors 可以用来重新激活奇偶行颜色设置。
\begin{table}[htbp]
\centering
\caption{colorless}
	\rowcolors{1}{white}{red}
	\begin{tabular}{lll}
		\hline
		操作系统 & 发行版 & 编辑器 \\
		Windows & MikTeX & TexMakerX \\
		Unix/Linux & teTeX & Kile \\
		Mac OS & MacTeX & TeXShop \\
		 通用 & TeX Live & TeXworks \\
		\hline
		\hiderowcolors
	\end{tabular}
\end{table}
	
\end{document}