\documentclass[UTF8]{article}

\usepackage{ctex}
\usepackage{amsmath}
\usepackage{amsthm}
\usepackage{amsfonts}
\usepackage{mathrsfs}
%\usepackage{caption}

\title{数学部分}
\author{Xmp}
\date{\today}

\begin{document}
\maketitle
\setcounter{tocdepth}{9}
\tableofcontents

\begin{abstract}
	为了使用\AmS-\LaTeX 提供的数学功能,需要在文档的序言部分加载amsmath宏包。
\end{abstract}

\section{数学模式}
行间模式和独立模式。

行间模式:$F=mg$

无编号独立公式:$$F=mg$$或者\begin{equation*}
	F=mg
\end{equation*}

有编号独立公式:\begin{equation}
	F=mg
\end{equation}

数学模式下也有加方框命令,例如:$E=mc^2$
\[E=mc^2\]
\[\boxed{E=mc^2}\]

\emph{建议不使用\$\$ \dots \$\$,因为它和\AmS-\LaTeX 有冲突。}

\section{基本元素}
\subsection{希腊字母}
比如:$\alpha \beta \gamma \tau \phi \chi$

\subsection{上下标和根号}
\[x_{ij}^2 \quad \sqrt{x} \quad \sqrt[3]{x} \]

\subsection{分数}
分数用\textbackslash frac命令表示,它会根据环境自动调整字号,比如在行间公式中小一点,在独立公式中则大一点。我们可以人工设置分数字号,比如\textbackslash dfrac命令把分数的字号设置为独立公式中的大小,而\textbackslash tfrac命令则把字号设为行间公式中的大小。
例如:$\frac{1}{2} \quad \dfrac{1}{2}$
\[\frac{1}{2} \tfrac{1}{2}\]

\subsection{运算符}
要用到时百度就好。
\newcommand{\myd}{\;\mathrm{d}}
\[\int x dx \quad \int x \myd x \]
多重积分符号:\[\int \quad \iint \quad \iiint \quad \iiiint \quad \idotsint \]

\subsection{箭头}
\[\xleftarrow{x+y+z}\quad \xrightarrow[x<y]{a*b*c}\]

\subsection{注音和音标}
例如:$\bar{x} \quad \vec{x} \quad \mathring{x}$等等。

\subsection{分隔符}
例如:$\overline{xxx} \quad \overleftrightarrow{xxxx}\quad \underbrace{xxx}$等等。
我们可以在上述分隔符前面加\textbackslash big \textbackslash Big \textbackslash bigg \textbackslash Bigg等命令来调整其大小。\LaTeX 原有的方法是在分隔符前面加\textbackslash left \textbackslash right来自动调整大小,但是效果不佳,所以amsmath不推荐用这种方法。
\[ \Bigg(\bigg(\Big(\big((x)\big)\Big)\bigg)\Bigg)\quad\Bigg[\bigg[\Big[\big[[x]\big]\Big]\bigg]\Bigg]\quad\Bigg\{\bigg\{\Big\{\big\{\{x\}\big\}\Big\}\bigg\}\Bigg\}\]\[\Bigg\langle\bigg\langle\Big\langle\big\langle \rangle\big\rangle\Big\rangle\bigg\rangle\Bigg\rangle\quad\Bigg\lvert\bigg\lvert\Big\lvert\big\lvert\lvert x\rvert\big\rvert\Big\rvert\bigg\rvert\Bigg\rvert\quad\Bigg\lVert\bigg\lVert\Big\lVert\big\lVert\lVert x\rVert\big\rVert\Big\rVert\bigg\rVert\Bigg\rVert \]

\subsection{省略号}
\[x_1,x_2,\dots,x_n\quad 1,2,\cdots,n\quad \vdots\quad\ddots\]

\subsection{空白间距}
就\, 用\: 这\; 个\quad 来\qquad 测试\! 每个空白!

\section{矩阵}
数学模式下可以用array环境)来生成矩阵,它提供了外部对齐和列对齐的控制参数。外部对齐是指整个矩阵和周围对象的纵向关系,有三种方式:居顶、居中(缺省)、居底,分别用t,c,b来表示;列对齐也有三种方式:居左、居中、居右,分别用l,c,r表示。\textbackslash \textbackslash 和\&用来分隔行和列。其语法如下:
%\begin{array}[外部对齐]{列对齐}
%	行列内容
%\end{array}
\[
\begin{array}{ccc}
x_1 & x_2 & \dots \\
x_3 & x_4 & \dots \\
\vdots & \vdots & \ddots
\end{array}
\]
amsmath的pmatrix,bmatrix,Bmatrix,vmatrix,Vmatrix等环境可以在矩阵两边加上各种分隔符,但是它们没有对齐方式参数。\textbackslash smallmatrix命令可以生成行间矩阵。
\[
\begin{pmatrix}a&b\\c&d \end{pmatrix} \quad
\begin{bmatrix}a&b\\c&d \end{bmatrix} \quad
\begin{Bmatrix}a&b\\c&d \end{Bmatrix} \quad
\begin{vmatrix}a&b\\c&d \end{vmatrix} \quad
\begin{Vmatrix}a&b\\c&d \end{Vmatrix} \quad
\]

Marry has a little matrix$(\begin{smallmatrix}a&b\\c&d\end{smallmatrix})$

\section{多行公式}
\subsection{长公式}
无对齐长公式:\begin{multline}
	x = a+b+c{}\\
	d+e+f+f
\end{multline}
对齐长公式:\[\begin{split}
	x={}&a+b+c+{}\\
	&d+e+f+g
\end{split}\]

\subsection{公式组}
无对齐公式组:\begin{gather}
	a=b+c+d\\
	x=y+z
\end{gather}
对齐公式组:
\begin{align}
	a &=b+c+d\\x &=y+2
\end{align}

\subsection{分支公式}
\[
y = \begin{cases}
	-x,\quad x\leq 0\\x,\quad x >0
\end{cases}
\]

\section{定理和证明}
\textbackslash newtheorem命令可以用来定义定理之类的环境,其语法如下。

语法:{环境名}[编号延续]{显示名}[编号层次]

下面的代码定制了四个环境:定义、定理、引理和推论,它们都在一个section内统一编号,而引理和推论会延续定理的编号。定义环境后,再使用。
\newtheorem{definition}{定义}[section]
\newtheorem{theorem}{定理}[section]
\newtheorem{lemma}[theorem]{引理}
\newtheorem{corollary}[theorem]{推论}
\begin{definition}
	Java是一种跨平台的编程语言
\end{definition}
\begin{theorem}
咖啡因可以刺激人的中枢神经
\end{theorem}
\begin{lemma}
茶和咖啡都会使人的大脑兴奋。
\end{lemma}
\begin{corollary}
	晚上喝咖啡可能会导致失眠
\end{corollary}
amsthm宏包提供proof环境,证明:
\begin{proof}[命题物质无限可分的证明]一尺之棰,日取其半,万世不竭\end{proof}。

\section{数学字体}
和文本模式类似,我们在数学模式下也可以选用不同的字体样式)。\textbackslash mathbb和\textbackslash mathfrak需要amsfonts宏包,\textbackslash mathscr需要mathrsfs宏包。
$$\mathrm{ABCXYZ}$$
\[\mathsf{ABCXYZ}\]
$$\mathtt{ABCXYZ}$$
\[\mathcal{ABCXYZ}\]
$$\mathbf{ABCXYZ}$$
\[\mathit{ABCXYZ}\]
\[\mathbb{ABCXYZ}\]
$$\mathfrak{ABCXYZ}$$
\[\mathscr{ABCXYZ}\]

\end{document}