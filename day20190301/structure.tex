\documentclass[UTF8]{article}

\usepackage{ctex}
\usepackage{marginnote}
\usepackage{listings}
\usepackage{xcolor}
\usepackage{hyperref}
%\usepackage{natbib}

\title{结构}
\author{Xmp}
\date{\today}

\begin{document}
\maketitle
\setcounter{tocdepth}{7}
\tableofcontents
\begin{abstract}
	\begin{quote}
		善张网者引其纲,不一一摄万目而后得。
		
		\marginnote{---《韩非子·外储说右》}
	\end{quote}	
\end{abstract}

\lstset
{
	numbers=left, 
	numberstyle= \tiny, 
	keywordstyle= \color{ blue!70},
	commentstyle= \color{red!50!green!50!blue!50}, 
	frame=shadowbox, % 阴影效果
	rulesepcolor= \color{ red!20!green!20!blue!20} ,
	escapeinside=``, % 英文分号中可写入中文
	xleftmargin=2em,xrightmargin=2em, aboveskip=1em,
	framexleftmargin=2em
}

\section{长文档}
在主控文档中引用子文档。使用\textbackslash include命令会新起一页,如果不想要新页可以改用\textbackslash input命令。

\begin{lstlisting}
\begin{document}
\include{chapter1.tex}
\include{chapter2.tex}
...
\end{document}
\end{lstlisting}

当文档很长时,可以使用syntonly宏包,这样编译时就只检查语法,而不生成结果文件。

\begin{lstlisting}
\usepackage{syntonly}
...
\syntaxonly
\end{lstlisting}

\section{标题}
我们用\textbackslash title,\textbackslash author,\textbackslash date等命令分别设置标题、作者、日期。标准文档类没有为作者所属单位定义专门的命令,我们可以把单位直接写在作者下面,作者和单位之间的联系通常用一些特殊符号来标注。设置上述内容之后,我们用 \maketitle 命令来输出它们。

文档类选项notitlepage和titlepage可以用来控制标题是否单独占一页。report和book文档类中的标题缺省独占一页,article文档类中的标题缺省和正文等混居一页。

\section{目录}
\textbackslash tableofcontents命令可以生成整个文档的章节目录。\textbackslash setcounter命令指定目录的层次深度。如果不想让某个章节标题出现在目录里,则可以使用带$*$的命令来声明章节。

\begin{lstlisting}
\chapter*{...}
\section*{...}
...
\end{lstlisting}

\textbackslash listoffigures和\textbackslash listoftables命令来生成图目录和表目录。

当章节或图表等结构发生变化时,我们需要执行两次编译命令以获得正确的目录。

另外我们也可以利用caption 宏包来自定义类似于插图和表格的浮动环境。本文中的例子浮动环境第二行代码指定loe为例目录中间文件后缀,example为环境名,例为浮动环境标题前缀,例目录是目录的标题。定义该环境后,其使用方法与插图、表格浮动环境一样。

\begin{lstlisting}
语法:\DeclareCaptionType[选项]{环境}[名称][目录名]

\DeclareCaptionType[fileext=loe]{example}
		[例][例目录]
\begin{example}[h]
...
\end{example}
\end{lstlisting}

\section{参考文献}
\subsection{thebibliography}
\LaTeX 中最原始的方法是用thebibliography 环境和\textbackslash bibtem命令来定义参考文献条目。第一行的参数9是参考文献条目编号的宽度;如果有几十个条目,可以把该参数改为99。
\begin{thebibliography}{9}
\bibitem{Rowling_1997}
	Joanne K. Rowling,
	\emph{Harry Potter and the Philosopher's Stone}.
	Bloomsbury, London,
	1997.
\end{thebibliography}
thebibliography环境一般放在文档末尾。定义了参考文献之后,可以用\textbackslash cite命令在正文中\cite{Rowling_1997}引用条目。

\subsection{BibTex}
用数据
库文件.bib 记录参考文献条目,用样式文件.bst 设置显示格式。普通用户一般不需要改动样式文件,只须维护数据库。

使用专门的文献管理软件提高效率,如JabRef。含有参考文献的文档更麻烦,它需要依次执行四次编译操作。建议用统一的样式。编译时用 xelatex 编译主控文档,而用 bibtex 编译各个子文档。

\begin{lstlisting}
xelatex master(.tex)
bibtex chapter1(.tex)
bibtex chapter2(.tex)
...
xelatex master(.tex)
xelatex master(.tex)
\end{lstlisting}

\subsection{Natbib}
natbib宏包支持两种模式:作者-年份和数字。提供三种列表样式:plainnat,abbrvnat,unsrtnat.

\section{索引}
makeidx宏包提供了索引功能。应用它时,我们首先要在文档序言部分引用宏包,并使用\textbackslash makeindex 命令;其次在正文中需要索引的地方定义索引,注意索引关键字在全文中须保持唯一;最后在合适的地方 (一般是文档末尾) 打印索引。

\begin{lstlisting}
\usepackage{makeidx}
\makeindex
...
\begin{document}
\index{索引关键字}
...
\printindex
\end{document}	
\end{lstlisting}

当编译含索引的文档时,用户需要执行三次编译操作。

\section{超链接}
hyperref宏包。

\label{sec:hyperlink}
编号形式:\ref{sec:hyperlink}
文字形式:\hyperref[sec:hyperlink]{链接}
\url{www.baidu.com}
\href{www.baidu.com}{百度}

\section{结构名}









\end{document}