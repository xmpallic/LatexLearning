\documentclass[UTF8]{article}

\usepackage{ctex}
\usepackage{paralist}

%\renewcommand{labelitemi}{-}
%\renewcommand{\theenumi}{\alph{enumi}}

\begin{document}

%基本列表:有序列表、无序列表、描述列表
\begin{itemize}
	\item C++
	\item Java
	\item HTML
\end{itemize}

\begin{enumerate}
	\item C++
	\item Java
	\item HTML
\end{enumerate}	
	
\begin{description}
	\item[C++] 编程语言
	\item[Java] 编程语言
	\item[HTML] 标记语言
\end{description}	

%其它列表,使用paralist宏包,节省空间
%压缩列表
\begin{compactitem}
	\item C++
	\item Java
	\item HTML
\end{compactitem}

\begin{compactenum}
	\item C++
	\item Java
	\item HTML
\end{compactenum}	

\begin{compactdesc}
	\item[C++] 编程语言
	\item[Java] 编程语言
	\item[HTML] 标记语言
\end{compactdesc}	

%行间列表
\begin{inparaitem}
	\item C++
	\item Java
	\item HTML
\end{inparaitem}

\begin{inparaenum}
	\item C++
	\item Java
	\item HTML
\end{inparaenum}

\begin{inparadesc}
	\item[C++] 编程语言
	\item[Java] 编程语言
	\item[HTML] 标记语言
\end{inparadesc}

%定制列表
\renewcommand{\labelitemi}{-}
\renewcommand{\theenumi}{\alph{enumi}}
\begin{itemize}
	\item C++
	\item Java
	\item HTML
\end{itemize}

\begin{enumerate}
	\item C++
	\item Java
	\item HTML
\end{enumerate}

\subsection{盒子}
\subsubsection{初级盒子}
\mbox{010 6278 5001}%把一组对象组合起来
\fbox{010 6278 5001}%在此基础上加个边框
\subsubsection{中级盒子}
%稍复杂的 \makebox 和 \framebox 命令提供了宽度和对齐方式控制的选项。其对齐方式有居中 (缺省) 、居左、居右和分散对齐,分别用 c, l, r, s 来表示。
%语法:[宽度][对齐方式]{内容}
\makebox[100pt][c]{仪仗队}
\framebox[100pt][s]{仪仗队}

\subsubsection{高级盒子}
%大一些的对象比如整个段落可以用 \parbox 命令或 minipage 环境,两者语法类似,有宽度、高度、外部对齐、内部对齐等选项。这里的外部对齐是指该盒子与周围对象的纵向关系,有三种方式:居顶、居中和居底对齐,分别用 t, c, b 来表示。内部对齐是指该盒子内部内容的纵向排列方式,也是同样三种。
%语法:[外部对齐][高度][内部对齐]{宽度}{内容}
\fbox{%
	\parbox[c][36pt][t]{170pt}{
	锦瑟无端五十弦,一线一柱思华年。庄生晓梦迷蝴蝶,望帝春心托杜鹃。
	}%
}
\hfill
\fbox{%
	\begin{minipage}[c][36pt][b]{170pt}
	沧海月明珠有泪,蓝田玉暖玉生烟。此情可待成追忆,只是当时已惘然。	
	\end{minipage}%
}

\subsection{交叉引用cross reference}
需要编译两次,一般。
%marker为标签名,它在全文中要保持唯一
一个标签\label{marker}
...
第\pageref{marker}页\ref{marker}节




























	
\end{document}