\documentclass[UTF8]{article}

\usepackage{ctex}

\title{Fonts}
\author{Xmp}
\date{\today}

\begin{document}
\maketitle
\tableofcontents
\newpage
\begin{abstract}
英文中的 typeface 和 font 一般都被翻译为字体,传统印刷业通常使用typeface,电脑字体通常使用 font;当然也有很多人混用这两个词。电脑字体的诸多相关概念可以划分为三个层次:
1. 编码层,字符 (包括字母、数字、符号、控制码等) 的索引和编码,也就是字符集 (character set) 和字符编码 (character encoding)。
2. 格式层,字形 (glyph) 的定义描述方法,以及字体的文件存储格式。
3. 显示层,字体的外在表现形式,比如字体的样式,或具体的字体。
\end{abstract}

\section{字符集和编码}	
众所周知电脑内部采用二进制编码,因为它易于用电子电路实现。所有字符在电脑内部都是用二进制表示的,字符集的二进制编码被称为字符编码,有时人们也会混用这两个术语。	

\section{字体格式}
\subsection{点阵和矢量字体}
电脑字体的数据格式可以分为三大类:点阵(bitmap)字体、轮廓(outline)字体和笔画 (stroke-based) 字体。

点阵字体通过点阵来描述字形。早期的电脑受到容量和绘图速度的限
制,多采用点阵字体。点阵字体后来渐渐被轮廓字体所取代,但是很多小字
号字体仍然使用它,因为这种情况下轮廓字体缩放太多会导致笔画不清晰。
轮廓字体又称作矢量字体,它通过一组直线段和曲线来描述字形。轮
廓字体易于通过数学函数进行缩放等变换,形成平滑的轮廓。轮廓字体的
主要缺陷在于它所采用的贝塞尔曲线 (Bézier curves) 在光栅设备 (比如显示器
和打印机) 上不能精确渲染,因而需要额外的补偿处理比如字体微调 (font
hinting) 。但是随着电脑硬件的发展,人们一般不在意它比点阵字体多出的
处理时间。

笔画字体其实也是轮廓字体,不过它描述的不是完整的字形,而是笔
画。它多用于东亚文字。	

\subsection{常见字体格式}
当前常见的轮廓字体格式有:Type 1、TrueType、OpenType。	
	
\subsection{合纵连横}
当年 Adobe推出 Type 1 和 Type 3 时,前者收费,后者是公开的自由规
范。Type 1 专利许可费十分昂贵,穷人们只好用免费的 Type 3。为了打破这
种垄断,Apple 开发了 TrueType。1991 年 TrueType 发布之后,Adobe 随即公
开了 Type 1 的规范,它从贵族堕落为平民,因而流行开来。

1980 年代中后期,Adobe 的大部分盈利来自于 PostScript 解释器的许可
费。面对这种垄断局面,微软和 Apple 联合了起来。微软把买来的 PostScript
解释器 TrueImage 授权给 Apple,Apple 则把 TrueType 授权给微软。

微软得陇望蜀,又企图获得 AAT 的许可证,未遂。为了打破 Apple 的垄
断,微软联合 Adobe 在 1996 年发布了 OpenType。Adobe 在 2002 年末将其字
体库全面转向 OpenType。
	
\section{常见字体}
\section{字体的应用}	
\subsection{早期技术}
\subsubsection{latex 和 DVI}
\subsubsection{XeTex}

\section{中文解决方案}
没有解释的段落个人觉得对于目前的学习状态不需要记录。	
	
	
	
	
	
	
	
	
	
	
	
	
	
	
	
	
	
	
	
	
	
	
	
	
	
	
	
	
	
\end{document}